% SIAM Shared Information Template
% This is information that is shared between the main document and any
% supplement. If no supplement is required, then this information can
% be included directly in the main document.


% Packages and macros go here
\usepackage{lipsum}
\usepackage{amsfonts}
\usepackage{graphicx}
\usepackage{epstopdf}
\usepackage{algorithmic}
\ifpdf
  \DeclareGraphicsExtensions{.eps,.pdf,.png,.jpg}
\else
  \DeclareGraphicsExtensions{.eps}
\fi

% Add a serial/Oxford comma by default.
\newcommand{\creflastconjunction}{, and~}

% Used for creating new theorem and remark environments
\newsiamremark{remark}{Remark}
\newsiamremark{hypothesis}{Hypothesis}
\crefname{hypothesis}{Hypothesis}{Hypotheses}
\newsiamthm{claim}{Claim}


%\newcommand{\TheTitle}{An adaptive partition of unity method for Chebyshev polynomial interpolation on Hypercubes} 
%\newcommand{\TheAuthors}{Kevin W. Aiton, Tobin A. Driscoll}


% Sets running headers as well as PDF title and authors
\headers{Adaptive partition of unity}{Kevin W. Aiton, Tobin A. Driscoll}

% Title. If the supplement option is on, then "Supplementary Material"
% is automatically inserted before the title.
\title{An adaptive partition of unity method for Chebyshev polynomial interpolation on Hypercubes \thanks{Submitted to the editors May 1, 2018.
\funding{This research was supported by National Science Foundation grant DMS-1412085.}}}

% Authors: full names plus addresses.
\author{Kevin W. Aiton, Tobin A. Driscoll}

\usepackage{amsopn}
\DeclareMathOperator{\diag}{diag}


%%% Local Variables: 
%%% mode:latex
%%% TeX-master: "ex_article"
%%% End: 
