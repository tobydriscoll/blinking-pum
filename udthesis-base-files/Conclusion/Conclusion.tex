\chapter{Conclusion}

In Chapter~\ref{pu_1d} we create a new method that offers a simple way to adaptively construct infinitely smooth approximations of functions that are given explicitly or that solve BVPs. By recursively constructing the PU weights with the binary tree, we avoid the need to determine the neighbors of each patch. We extend the tree based approximations to arbitrary dimension in Chapter~\ref{pu_nd}, allowing us to construct approximations that are adapted to the features of the function but still infinitely smooth. We extended this technique to arbitrary domains via the least squares technique. 

In Chapter~\ref{snk_chap} we described a framework for overlapping domain decomposition in which overlap regions are discretized independently by the local subdomains, even in the formulation of the global problem. Communication between subdomains occurs only via interpolation of values to interface points. This formulation makes it straightforward to apply high-order or spectral discretization methods in the subdomains and to adaptively refine them. The technique may be applied to precondition a linearized PDE, but it may also be used to precondition the nonlinear problem before linearization, to get what we call the Schwarz--Newton--Krylov (SNK) technique. In doing so, one gets the same benefit of faster Krylov inner convergence, but the resulting nonlinear problem is demonstrably easier to solve in terms of outer iterations and robustness. 

We apply SNK to solve a blinking eye model in Chapter~\ref{chap_eye}. Domain decomposition allows us to achieve the necessary resolution to perserve the volume over a blink cycle in a way that is computationally efficient. We discovered through numerical experiments that the influx and outflux condtions (\ref{influx_fun},\ref{out_flux_fun}) could be causing the tear film thickness to become negative, producing unrealistic results. Further examination and experimentation is needed to validate the model.

